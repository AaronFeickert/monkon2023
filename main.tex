\documentclass[aspectratio=169]{beamer}

\usepackage{amsmath}
\usepackage{amssymb}
\newcommand{\com}{\mathsf{Com}}

\setbeamertemplate{footline}[frame number]
\beamertemplatenavigationsymbolsempty

\title{Spats: User-defined Confidential Assets and Tokens}
\author{Aaron Feickert, Ph.D.}
\date{Monero Konferenco \\ 23 June 2023}

\begin{document}

\frame{\titlepage}

\begin{frame}{Acknowledgments and disclaimer}
	Spats is joint work with \textbf{Aram Jivanyan} of Firo and Yerevan State University. \\~\\

	The author is head of research at \textbf{Cypher Stack}, and research for Spats was conducted for \textbf{Firo}. \\~\\

	The material in this presentation is for informational use only, has not undergone formal external review, and does not constitute professional advice of any kind.
\end{frame}

\begin{frame}{Overview}
	Protocols like \textbf{Spark} and \textbf{Seraphis} and \textbf{RingCT} already do neat tricks. They can:
	\begin{itemize}
		\item Dissociate recipient addresses from generated coin data
		\item Provide ambiguity about generated coin spend status
		\item Protect coin values and other metadata \\~\\
	\end{itemize}

	But they only support a \textbf{single asset type}. \\~\\

	We want to support:
	\begin{itemize}
		\item Multiple asset types that have independent supply
		\item Protection of asset types and values in transactions
		\item Non-fungible tokens whose identifiers are protected
	\end{itemize}
\end{frame}

\begin{frame}{Commitments}
	How do we represent value in a safe way? \\~\\

	A coin with value $v$ can be represented in part by a \textbf{commitment} that both hides and binds the value by including a random mask $m$. \\~\\

	We can write this using Pedersen commitments as $C = \com(v, m) = mG + vH$, where $G$ and $H$ are fixed group elements. \\~\\

	It's not possible to look at $C$ and extract the value or mask. \\~\\

	We can show that a transaction balances in zero knowledge using commitment sums and differences via a Schnorr proof without revealing values or masks.
\end{frame}

\begin{frame}{Multiple assets}
	But suppose we want to support \textbf{multiple asset types} with independent supply in a safe way. \\~\\

	Some asset types might be \textbf{fungible}, where \underline{coins} are divisible with no particular differentiation. \\~\\

	Others might be \textbf{non-fungible}, where \underline{tokens} have identifiers and must be differentiated. \\~\\

	In all cases, transactions should be as uniform as possible, and protect data and metadata until or unless the user chooses to reveal it.
\end{frame}

\begin{frame}{Spats}
	\textbf{Spats} is a protocol intended as an extension to confidential transaction protocols like Spark, Seraphis, or RingCT to meet these goals. \\~\\

	The basic idea is to extend value commitments to include additional data: an \textbf{asset type} $a$ and an \textbf{identifier} $\iota$.
	This is easy with Pedersen commitments. \\~\\

	All assets, whether fungible or non-fungible, have a unique asset type. \\~\\

	Fungible assets don't use identifiers, but each token of a non-fungible type has a unique identifier within its type.
\end{frame}

\begin{frame}{Commitments}
	Previously, a value commitment looked pretty simple:
	$$\com(v, m)$$

	Now, it looks a little different, with the form $\com(a, \iota, v, m)$:

	\begin{center}
		\setlength{\tabcolsep}{20pt}
		\renewcommand{\arraystretch}{1.5}
		\begin{tabular}{cc}
			\textbf{Fungible} & \textbf{Non-fungible} \\
			$\com(a, 0, v, m)$ & $\com(a, \iota, 1, m)$
		\end{tabular}
	\end{center}

	Fungible asset types always set the identifier $\iota = 0$ (more on this later). \\~\\

	Non-fungible asset types always set the value $v = 1$ for atomicity.
\end{frame}

\begin{frame}{Minting}
	Right now, a minted coin of value $v$ can include a Schnorr proof that its commitment is to the expected value, but without revealing the secret mask $m$.
	This protects supply. \\~\\

	With Spats, we simply extend this. \\~\\

	A minted coin or token reveals its \textbf{type}, \textbf{identifier}, and \textbf{value} publicly.
	This is important for ensuring supply is correct across types according to whatever consensus rules the protocol chooses. \\~\\

	This still supports dissociating minted assets from recipient addresses.
\end{frame}

\begin{frame}{Transactions}
	So how do transactions work? There's a Simple Rule: \\~\\

	\textbf{All consumed and generated assets in a transaction must share the same asset type and identifier.} \\~\\

	For fungible assets, a transaction might consume several coins of the same type, and generate several coins of the same type (including change).
	Remember, $\iota = 0$ here! \\~\\

	For non-fungible assets, the transaction will transfer only a single token; we can include zero-value dummy tokens for uniformity too.
	Remember, $v = 1$ here!
\end{frame}

\begin{frame}{The catch: fees}
	If you think about it for a while, there's a problem.

	\begin{center}
		\alert{How can you issue fees if the asset type is protected?}
	\end{center}

	The answer is that you can't, so Spats makes a shitty tradeoff by adding a second Simple Rule: \\~\\

	\textbf{Transactions include a ``fee side'' that consumes and generates coins of a fixed base asset type.} \\~\\

	This means a transaction really does two things:
	\begin{itemize}
		\item Issues fees in a fixed base asset type
		\item Transfers coins or tokens of an arbitrary protected asset type
	\end{itemize}
\end{frame}

\begin{frame}{The slightly gory details}
	The two Simple Rules need to be enforced cryptographically.
	We can do this by adding a few zero-knowledge proofs (ZKPs) into the mix. \\~\\

	\textbf{The Fee-Side ZKPs:}
	\begin{itemize}
		\item All consumed and generated coins on the fee side have $a = \iota = 0$
		\item The fee side balances
	\end{itemize}

	\textbf{The Asset-Side ZKPs:}
	\begin{itemize}
		\item All consumed and generated coins/tokens on the asset side have the same $a$ and $\iota$
		\item The asset side balances \\~\\
	\end{itemize}

	This asserts that minimal information is leaked.
	On the fee side, we need to know what asset is being used.
	On the asset side, we must \textbf{not} know what asset is being used.
\end{frame}

\begin{frame}{The slightly more gory details}
	Fortunately, we know how to build these zero-knowledge proofs! \\~\\

	Each is a Schnorr-type proof, but with nice properties:
	\begin{itemize}
		\item Size is independent of the number of coins/tokens involved
		\item Verification is efficient and can be batched
		\item There are no icky trusted setups
		\item Security proofs are straightforward \\~\\
	\end{itemize}

	They can be built using a power-of-challenge design. \\~\\

	We also need to modify range proofs to support vector commitments, but this is easy to do with Bulletproofs and Bulletproofs+ with only minor changes.
\end{frame}

\begin{frame}{Efficiency}
	Spats transactions require modifications to existing proofs, as well as new proofs.

	\begin{center}
		\setlength{\tabcolsep}{20pt}
		\begin{tabular}{lr}
			\textbf{Component} & \textbf{$\Delta$ (bytes)} \\
			\hline
			Range proof & 64 \\
			Balance (asset) & 128 \\
			Type (base) & 128 \\
			Type (asset) & 256 \\
			\hline
			Total & 576
		\end{tabular}
	\end{center}

	(The total does not include encrypted encodings of asset types and identifiers, which depend on protocol rules and transaction structure.) \\~\\

	New proofs are speedy, adding minimal verification complexity.
\end{frame}

\begin{frame}{Migration}
	Spats needs to play nicely with existing protocols like Spark or Seraphis or RingCT.
	You shouldn't need to migrate assets from one pool to another, since this is risky. \\~\\

	Fortunately, the design makes this a snap. \\~\\

	Because base assets require $a = \iota = 0$, this means \textbf{all existing coins} are automatically Spats base asset coins!
	That is, $\com(0, 0, v, m) = \com(v, m)$ holds between new and existing commitments. \\~\\

	Further, because the two Simple Rules are enforced using ZKPs on coin commitments, there is always a \textbf{single pool} of assets available for transactions, regardless of asset type! \\~\\

	This means that Spats supports clean migrations from existing protocols.
\end{frame}

\begin{frame}{Ownership proofs}
	Support for non-fungible tokens implies something a bit unique: \textbf{ownership proofs}. \\~\\

	It's likely that you'll need to prove ownership of a token, but this really involves showing several things:
	\begin{itemize}
		\item The token's commitment is to a given type $a$ and identifier $\iota$ (and value $v = 1$)
		\item You know the secret data required to transfer the token
		\item The proof isn't being replayed as a trick
		\item The token has not already been transferred elsewhere \\~\\
	\end{itemize}

	The first two are straightforward using Schnorr-type proofs. \\~\\

	The third is easy using Fiat-Shamir message binding.
\end{frame}

\begin{frame}{Ownership proofs}
	Showing that a token hasn't been transferred is much trickier. \\~\\

	One approach is to do the following:
	\begin{itemize}
		\item Reveal the \textbf{linking tag} associated to the token
		\item Prove that the linking tag is correct
		\item Assert that the linking tag does not appear in any subsequent transaction
	\end{itemize}
	
	Unfortunately, this sucks.
	Having the tag means the verifier can see if/when the token is later transferred, which is leaky. \\~\\

	A better way is to prove the tag doesn't appear on chain without revealing it, which requires more clever proof techniques and is work in progress.
\end{frame}

\begin{frame}{The gist}
	\textbf{Spats} is a new design for user-defined confidential assets and tokens with nice properties:
	\begin{itemize}
		\item Fungible and non-fungible asset types are supported
		\item Transactions can be made uniform
		\item Asset types, identifiers, and values are protected
		\item Fees can be denominated in a fixed base type
		\item Users can prove token ownership (in a leaky way)
		\item Proving systems are straightforward
		\item Migration is clean
		\item All coins and tokens can share a single pool
	\end{itemize}
\end{frame}

\begin{frame}{Questions?}
	\Large
	\begin{center}
		\begin{tabular}{ll}
			\textbf{Preprint} & \url{ia.cr/2022/288} \\
			& \url{github.com/AaronFeickert/spats} \\
			\\
			\textbf{Slides} & \url{github.com/AaronFeickert/monkon2023} \\
			\\
			\textbf{Email} & \texttt{aaron@cypherstack.com}
		\end{tabular}
	\end{center}
\end{frame}
	
\end{document}
